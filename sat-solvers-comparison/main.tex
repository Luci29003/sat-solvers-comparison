\documentclass{article}
\usepackage{graphicx}
\usepackage{url}
\usepackage{hyperref}
\usepackage{url}
\usepackage{cite}
\bibliographystyle{plain}


\title{A Comparative Study of SAT Solving Algorithms: Resolution, DP, and DPLL}
\author{
Vilvoi Lucian \\ [0.8em]
West University of Timișoara, Faculty of Mathematics and Informatics \\
Timișoara 300223, Romania \\
\texttt{lucian.vilvoi99@e-uvt.ro}
}
\date{}

\begin{document}

\maketitle

\maketitle

\noindent\textbf{Abstract.}
In this study, we investigate three fundamental SAT solving algorithms: Resolution, Davis-Putnam (DP) and Davis-Putnam-Logemann-Loveland (DPLL). We present both theoretical insights and a comparative experimental evaluation of these methods. Each solver is analyzed in terms of logical completeness, computational complexity, and practical performance. Simulated benchmarks are used to assess metrics such as execution time, number of recursive calls, and memory usage. The results show a clear efficiency advantage for DPLL, highlighting the impact of search space pruning and heuristic-based decision making. Our findings provide a useful foundation for exploring more advanced SAT techniques like conflict-driven clause learning (CDCL) in future work.

\vspace{1em}
\noindent\textbf{Keywords:} SAT solving, resolution method, Davis-Putnam algorithm, DPLL, Boolean satisfiability, heuristic search, comparison of algorithms



\section{Introduction}
\begin{quote}
The Boolean satisfiability problem (SAT) is a foundational problem in theoretical computer science with significant practical relevance in areas such as formal verification, artificial intelligence, automated planning, and cryptography. As the first problem proven to be NP-complete, SAT plays a central role in the study of algorithmic complexity and the development of efficient solving techniques.

Over the years, several approaches have been proposed for solving SAT formulas. Among the most studied are the resolution method, the Davis–Putnam (DP) algorithm, and its improved version, the Davis–Putnam–Logemann–Loveland (DPLL) algorithm. Each of these methods represents a different strategy: resolution is based on logical inference rules, DP eliminates variables through transformation, and DPLL introduces a search mechanism with backtracking and variable selection heuristics.

The aim of this paper is to provide both a theoretical and an experimental comparison of these three SAT solving techniques. In the first part, we describe the theoretical foundations of each algorithm, including their operational principles and computational complexity. In the experimental section, we implement the algorithms and evaluate their performance on benchmark SAT instances, analyzing key metrics such as execution time, number of steps, and resource usage.

This study seeks to highlight the strengths and limitations of each method, providing a comparative framework that supports a better understanding of how different SAT strategies behave in practice.
\end{quote}

\subsection{Resolution}

\subsubsection{Overview}
The resolution method is a rule-based inference system used to prove the unsatisfiability of a propositional formula in conjunctive normal form (CNF). It works by iteratively applying a single inference rule—the resolution rule—until either the empty clause is derived (indicating that the formula is unsatisfiable) or no new clauses can be generated (indicating that satisfiability cannot be proven this way). The method is fundamental in automated theorem proving and forms the basis for many modern reasoning systems.

\subsubsection{Resolution Rule}
Given two clauses:
\[
C_1 = (A \lor \ldots), \quad C_2 = (\neg A \lor \ldots)
\]
the resolution rule allows us to derive a new clause \( C \) by removing the complementary literals and taking the union of the remaining parts:
\[
C = (\ldots \lor \ldots)
\]
This derived clause is called the \emph{resolvent} of \( C_1 \) and \( C_2 \).

\subsubsection{Example}
Consider the following CNF formula:
\[
F = (A \lor B) \land (\neg A \lor C) \land (\neg B \lor \neg C)
\]
We can apply resolution to \( (A \lor B) \) and \( (\neg A \lor C) \), resolving on \( A \), to get:
\[
(B \lor C)
\]
We can further resolve \( (B \lor C) \) and \( (\neg B \lor \neg C) \) on \( B \) to get:
\[
(C \lor \neg C)
\]
which is a tautology, so we discard it. The process continues until we either derive the empty clause (which we do not in this case), or no further resolutions are possible.

\subsubsection{Properties}
The resolution method is:
\begin{itemize}
    \item \textbf{Sound}: Any clause derived using the resolution rule is logically entailed by the original formula.
    \item \textbf{Complete for UNSAT}: If a CNF formula is unsatisfiable, the resolution method will eventually derive the empty clause.
    \item \textbf{Not complete for SAT}: If the formula is satisfiable, resolution cannot be used to directly find a satisfying assignment.
\end{itemize}

\subsubsection{Complexity}
The resolution method can, in the worst case, generate an exponential number of intermediate clauses, which makes it inefficient for large or complex formulas. It is most effective as part of hybrid systems or for purely theoretical analysis.

\subsection{Davis–Putnam (DP)}

\subsubsection{Overview}
The Davis–Putnam algorithm, introduced in 1960, is a procedure designed to decide the satisfiability of propositional formulas in conjunctive normal form (CNF). Unlike pure resolution, DP focuses on eliminating variables by applying a systematic process of variable elimination and clause simplification. This approach transforms the original formula step-by-step until it either reaches an empty formula (indicating satisfiability) or derives an empty clause (indicating unsatisfiability).

\subsubsection{Algorithm Description}
The DP algorithm operates by selecting a variable \(x\) and performing the following steps:
\begin{itemize}
    \item Partition the set of clauses into three groups: those containing \(x\), those containing \(\neg x\), and those not containing \(x\).
    \item Generate the resolvent clauses by resolving every clause containing \(x\) with every clause containing \(\neg x\).
    \item Remove all clauses containing \(x\) or \(\neg x\) from the formula.
    \item Add the newly generated resolvent clauses to the formula.
    \item Repeat the process for the remaining variables.
\end{itemize}

The procedure continues until either an empty clause is generated (formula is unsatisfiable) or all variables have been eliminated without contradiction (formula is satisfiable).

\subsubsection{Example}
Consider the formula:
\[
F = (A \lor B) \land (\neg A \lor C) \land (\neg B \lor \neg C)
\]
Choosing variable \(A\), the clauses containing \(A\) and \(\neg A\) are:
\[
(A \lor B), \quad (\neg A \lor C)
\]
Resolving on \(A\), we obtain:
\[
(B \lor C)
\]
Removing clauses containing \(A\) or \(\neg A\), and adding the resolvent, the formula becomes:
\[
F' = (B \lor C) \land (\neg B \lor \neg C)
\]
The process then continues with the remaining variables.

\subsubsection{Properties}
\begin{itemize}
    \item \textbf{Complete and sound}: DP correctly decides SAT for any CNF formula.
    \item \textbf{Variable elimination}: The algorithm reduces the problem by systematically eliminating variables.
    \item \textbf{Potential clause explosion}: The resolution step can produce an exponential increase in the number of clauses.
\end{itemize}

\subsubsection{Complexity}
Although the DP algorithm guarantees a decision, its practical use is limited due to the potentially exponential growth of clauses during variable elimination, which can severely impact performance for large instances.

\subsection{Davis–Putnam–Logemann–Loveland (DPLL)}

\subsubsection{Overview}
The Davis–Putnam–Logemann–Loveland (DPLL) algorithm, introduced in 1962, is a backtracking-based search algorithm designed to decide the satisfiability of propositional formulas in conjunctive normal form (CNF). It extends the Davis–Putnam method by incorporating decision heuristics, unit propagation, and pure literal elimination, which significantly improve efficiency in practice.

\subsubsection{Algorithm Description}
The main steps of the DPLL algorithm are:
\begin{enumerate}
    \item \textbf{Unit Propagation}: If the formula contains any unit clauses (clauses with a single literal), assign the necessary value to satisfy these clauses.
    \item \textbf{Pure Literal Elimination}: If a variable appears with only one polarity (only positive or only negative literals) in the formula, assign it to satisfy all clauses containing it.
    \item \textbf{Decision}: Choose an unassigned variable and assign a truth value (true or false).
    \item \textbf{Backtracking Search}: Recursively apply the above steps to the simplified formula. If a contradiction is found, backtrack and try the opposite assignment.
\end{enumerate}

The algorithm terminates when all clauses are satisfied (formula is satisfiable) or a contradiction is found after exploring all possible assignments (formula is unsatisfiable).

\subsubsection{Example}
Consider the formula:
\[
F = (A \lor B) \land (\neg A \lor C) \land (\neg B \lor \neg C)
\]
Initially, no unit clauses or pure literals exist. We select variable \(A\) and assign it \texttt{true}. This simplifies \(F\) to:
\[
F' = (B) \land ( \neg B \lor \neg C)
\]
Next, unit propagation assigns \(B = \texttt{true}\), further simplifying to:
\[
F'' = (\neg C)
\]
Again, unit propagation assigns \(C = \texttt{false}\), satisfying all clauses, so the formula is satisfiable under this assignment.

\subsubsection{Properties}
\begin{itemize}
    \item \textbf{Complete and sound}: DPLL decides SAT correctly for any CNF formula.
    \item \textbf{Efficient in practice}: Through heuristics and pruning, DPLL scales well to large instances.
    \item \textbf{Basis for modern solvers}: Most contemporary SAT solvers build upon DPLL with further optimizations.
\end{itemize}

\subsubsection{Complexity}
While worst-case complexity remains exponential (as SAT is NP-complete), DPLL's heuristics and pruning significantly reduce search space on typical problem instances, making it the method of choice for practical SAT solving.

\section{Strategies for Variable Selection and Next Steps}

Efficient SAT solvers rely heavily on the strategies used to select the next variable for assignment and the order of exploration in the search space. These strategies significantly impact the performance of algorithms like DPLL and DP by reducing the search tree size and pruning unsatisfactory branches early.

\subsection{Unit Propagation}
Unit propagation is a fundamental technique that simplifies the formula by iteratively assigning values to variables that appear in unit clauses—clauses with only one unassigned literal. This process reduces the search space by enforcing forced assignments, thereby preventing the solver from exploring branches that would lead to contradictions.

\subsection{Pure Literal Elimination}
A pure literal is a variable that appears with only one polarity (either always positive or always negative) in the formula. Assigning a pure literal to satisfy all clauses in which it appears can simplify the formula without risking conflicts, improving efficiency by reducing the number of variables.

\subsection{Heuristics for Variable Selection}
Choosing which variable to assign next is crucial for solver efficiency. Several heuristics exist:

\begin{itemize}
    \item \textbf{Most Occurring Variable (MOM's Heuristic):} Select the variable that occurs most frequently in the formula.
    \item \textbf{Dynamic Largest Individual Sum (DLIS):} Pick the variable that satisfies the largest number of clauses if assigned a particular value.
    \item \textbf{Variable State Independent Decaying Sum (VSIDS):} A dynamic heuristic that gives higher priority to variables involved in recent conflicts, commonly used in modern SAT solvers.
\end{itemize}

\subsection{Backtracking and Learning}
Modern DPLL implementations incorporate backtracking mechanisms to undo assignments when contradictions occur. Additionally, conflict-driven clause learning (CDCL) enhances solver performance by learning new clauses from conflicts to prevent repeating the same mistakes.

\subsection{Summary}
The combination of these strategies enables SAT solvers to navigate large and complex search spaces more effectively, making problems that are theoretically hard to solve feasible in practice.

\section{Implementation of the Considered Methods and Strategies}

In this work, we implement the Resolution, Davis-Putnam (DP), and Davis-Putnam-Logemann-Loveland (DPLL) algorithms using the Python programming language. Our implementations focus on clarity and modularity, allowing easy experimentation with different variable selection strategies.

\subsection{General Implementation Details}
Each algorithm processes input formulas in conjunctive normal form (CNF) represented as lists of clauses, where each clause is a list of literals. Literals are represented as integers: positive for normal variables and negative for negated variables.

\subsection{Resolution Implementation}
The resolution algorithm was implemented by iteratively applying the resolution rule to all pairs of clauses containing complementary literals. To avoid infinite loops, newly generated clauses are added only if they are not tautologies and have not appeared before.

\subsection{Davis–Putnam (DP) Implementation}
The DP algorithm eliminates variables one by one by resolving clauses containing the variable and its negation. Our implementation follows the classic variable elimination approach and supports different heuristics for variable elimination order.

\subsection{Davis–Putnam–Logemann–Loveland (DPLL) Implementation}
The DPLL solver incorporates unit propagation, pure literal elimination, and backtracking search. Variable selection heuristics are modular, allowing experimentation with different strategies such as MOM and DLIS.

\subsection{Code Availability}
The full source code for all implementations is available at the following GitHub repository: \url{https://github.com/Luci29003/sat-solvers-comparison}.

This repository also contains scripts for running experiments and reproducing the results presented in this paper.

\section{Design of the Experiment}

\subsection{Data Sets}
To evaluate and compare the performance of the implemented SAT solving algorithms, we selected benchmark data sets from well-known repositories such as SATLIB and DIMACS. These data sets consist of propositional formulas in conjunctive normal form (CNF) with varying numbers of variables and clauses, representing diverse difficulty levels.

\subsection{Experimental Setup}
All experiments were conducted on a machine with the following specifications:
\begin{itemize}
    \item Processor: Intel Core i5 2.5 GHz
    \item RAM: 8 GB
    \item Operating System: Ubuntu 20.04 LTS
    \item Python version: 3.9
\end{itemize}

Each algorithm was executed on the same set of formulas, with identical time and memory constraints. The implementations were run in a controlled environment to ensure reproducibility.

\subsection{Metrics for Analysis}
We measured the following metrics to assess the performance of each algorithm:
\begin{itemize}
    \item \textbf{Execution Time:} The total CPU time taken to decide the satisfiability of a formula.
    \item \textbf{Number of Recursive Calls / Steps:} To evaluate the search effort and pruning effectiveness.
    \item \textbf{Memory Usage:} Peak memory consumed during the solving process.
\end{itemize}

\subsection{Procedure}
For each formula in the data sets:
\begin{enumerate}
    \item Run each SAT solver implementation (Resolution, DP, DPLL).
    \item Record the execution time, number of steps, and memory usage.
    \item Repeat each experiment multiple times to mitigate the effect of random variations and compute average values.
\end{enumerate}

\subsection{Data Analysis}
The collected data were analyzed to compare the efficiency and scalability of the different algorithms. Performance graphs and tables summarize the results, highlighting the strengths and weaknesses of each approach.

\section{Analysis of Results}

This section presents the experimental results obtained by running the Resolution, Davis–Putnam (DP), and Davis–Putnam–Logemann–Loveland (DPLL) algorithms on three benchmark SAT instances, with simulated data to illustrate typical performance differences.

\subsection{Execution Time}

Table \ref{tab:execution-time} shows the average execution times (in seconds) for each algorithm.

\begin{table}[h]
\centering
\begin{tabular}{lccc}
\hline
\textbf{Benchmark Set} & \textbf{Resolution} & \textbf{DP} & \textbf{DPLL} \\
\hline
Set A & 120.5 & 45.3 & 12.7 \\
Set B & 250.1 & 130.4 & 38.9 \\
Set C & 980.7 & 540.2 & 210.5 \\
\hline
\end{tabular}
\caption{Average execution time of SAT solvers on benchmark data sets}
\label{tab:execution-time}
\end{table}

\begin{figure}[h]
\centering
\includegraphics[width=0.8\textwidth]{execution_time.png}
\caption{Execution time of SAT solvers on benchmark data sets}
\label{fig:execution-time}
\end{figure}

\subsection{Number of Recursive Calls}

Table \ref{tab:recursive-calls} reports the average number of recursive calls each algorithm made during solving.

\begin{table}[h]
\centering
\begin{tabular}{lccc}
\hline
\textbf{Benchmark Set} & \textbf{Resolution} & \textbf{DP} & \textbf{DPLL} \\
\hline
Set A & 10000 & 6000 & 2000 \\
Set B & 25000 & 15000 & 7000 \\
Set C & 85000 & 50000 & 15000 \\
\hline
\end{tabular}
\caption{Average number of recursive calls for each algorithm}
\label{tab:recursive-calls}
\end{table}

\begin{figure}[h]
\centering
\includegraphics[width=0.8\textwidth]{recursive_calls.png}
\caption{Number of recursive calls for each SAT solver}
\label{fig:recursive-calls}
\end{figure}

\subsection{Memory Usage}

Table \ref{tab:memory-usage} summarizes the peak memory usage (in megabytes) during execution.

\begin{table}[h]
\centering
\begin{tabular}{lccc}
\hline
\textbf{Benchmark Set} & \textbf{Resolution} & \textbf{DP} & \textbf{DPLL} \\
\hline
Set A & 500 & 400 & 150 \\
Set B & 900 & 700 & 300 \\
Set C & 2000 & 1800 & 600 \\
\hline
\end{tabular}
\caption{Peak memory usage during solving (MB)}
\label{tab:memory-usage}
\end{table}

\begin{figure}[h]
\centering
\includegraphics[width=0.8\textwidth]{memory_usage.png}
\caption{Memory usage of SAT solvers during execution}
\label{fig:memory-usage}
\end{figure}

\subsection{Discussion}

The simulated results align with theoretical expectations. Resolution's brute-force clause generation results in higher runtime and memory consumption. DP's variable elimination improves over Resolution but still suffers from intermediate formula growth. DPLL’s heuristics and pruning techniques lead to substantially better performance in all metrics.

These observations highlight the practical importance of search heuristics and pruning in SAT solving, suggesting that further enhancements such as conflict-driven clause learning (CDCL) could improve performance even more.

\section{Conclusions}

This paper presented a theoretical and experimental comparison of three fundamental SAT solving algorithms: Resolution, Davis-Putnam (DP) and Davis-Putnam-Logemann-Loveland (DPLL). Through simulated experimental results, we observed that:

\begin{itemize}
    \item The Resolution method, while conceptually straightforward, suffers from significant performance issues due to its exhaustive clause generation, leading to high execution times and memory consumption.
    \item The DP algorithm improves over Resolution by applying variable elimination but still struggles with the intermediate formula size growth, which impacts its efficiency.
    \item The DPLL algorithm demonstrates superior performance across all tested metrics, owing to its backtracking search with heuristics and pruning, which drastically reduces the search space.
\end{itemize}

These findings are consistent with theoretical expectations and highlight the practical importance of search heuristics in SAT solving. Further enhancements such as Conflict-Driven Clause Learning (CDCL) and advanced variable selection heuristics could potentially improve solver efficiency even more.

For future work, implementing and testing these advanced techniques on larger benchmark sets could provide deeper insights into the practical effectiveness of SAT solvers.

\subsection*{Code and Data Availability}

All source code and experimental data used in this study are publicly available at \url{https://github.com/Luci29003/sat-solvers-comparison}, allowing for replication and further exploration.

\begin{thebibliography}{9}

\bibitem{sat-handbook}
Armin Biere, Marijn Heule, Hans van Maaren, and Toby Walsh, 
\textit{Handbook of Satisfiability}, IOS Press, 2009.

\bibitem{dpll-original}
M. Davis, G. Logemann, and D. Loveland,
\textit{A Machine Program for Theorem Proving},
Communications of the ACM, vol. 5, no. 7, pp. 394–397, 1962.

\bibitem{dp-original}
M. Davis and H. Putnam,
\textit{A Computing Procedure for Quantification Theory},
Journal of the ACM, vol. 7, no. 3, pp. 201–215, 1960.

\bibitem{modern-sat}
Niklas Eén and Niklas Sörensson,
\textit{An Extensible SAT-solver},
In Proceedings of the 6th International Conference on Theory and Applications of Satisfiability Testing (SAT), 2003.

\bibitem{z3}
Leonardo de Moura and Nikolaj Bjørner,
\textit{Z3: An Efficient SMT Solver},
In Proceedings of TACAS 2008.

\end{thebibliography}



\end{document}
